%-%-%-%-%-%-%-%-%-%-%-%-%-%-%-%-%-%-%-%-%-%-%-%-%-%
%  MC514 Sistemas Operacionais: Teoria e Pr�tica  %  
%  Projeto Final: Inicializa��o do Linux (Boot)   %
%  Data:26/06/2010                                %
%  Unicamp,Campinas,S�o Paulo,Brasil              %
%  Grupo:                                         %
%    - Alexandre Nobuo Kunieda                    %
%    - Daniel Catarino Biscalchin                 %
%    - Daniel Naoyoshi Fukuciro Parrode           %
%    - Tiago Chedraoui Silva                      %
%-%-%-%-%-%-%-%-%-%-%-%-%-%-%-%-%-%-%-%-%-%-%-%-%-%
\documentclass[dvips,11pt]{beamer}

%%% fontes %%%
\usepackage[T1]{fontenc}
\usepackage[latin9]{inputenc}
\usepackage[brazil]{babel}    % d� suporte para os termos na l�ngua portuguesa do Brasi


%%% matematicos %%%
\usepackage{amsmath}
\usepackage{amssymb}
\usepackage{mathptmx}

\usepackage{multicol}  
%%% figuras %%%
\usepackage{graphicx}
\usepackage{wrapfig}


%%% tabelas %%%
\usepackage{colortbl}
\usepackage{array}
\usepackage{longtable}
\usepackage{fancyvrb}
\usepackage{color}

%%% outros %%%
\usepackage{url}
\usepackage{textcomp}
\usepackage{hyperref} %internal links
\usepackage{color}       
\usepackage{indentfirst} %retira padrao americano de paragrafos
\usepackage{multicol}    
\numberwithin{table}{section}
\numberwithin{figure}{section} %numercao de figuras por secao


%%% usado para codigo %%%
\usepackage{listings}
\lstset{
  language = Java,
  basicstyle=\footnotesize\ttfamily, 
  numbers=left,               
  numberstyle=\tiny,         
  %stepnumber=2,              
  numbersep=5pt,             
  tabsize=2,                  
  extendedchars=true,  
  breaklines=true,       
  keywordstyle=\color{blue},
  frame=b,         
  stringstyle=\color{green}\ttfamily, 
  showspaces=false,          
  showtabs=false,             
  xleftmargin=17pt,
  framexleftmargin=17pt,
  framexrightmargin=5pt,
  framexbottommargin=4pt,
  %backgroundcolor=\color{lightgray},
  showstringspaces=false              
}
\lstloadlanguages{C}
\usepackage{caption}
\DeclareCaptionFont{white}{\color{white}}
\DeclareCaptionFont{green}{\color{green}}
\DeclareCaptionFormat{listing}{\colorbox[cmyk]{0.43, 0.35, 0.35,0.01}{\parbox{\textwidth}{\hspace{15pt}#1#2#3}}}
\captionsetup[lstlisting]{format=listing,labelfont=white,textfont=white, singlelinecheck=false, margin=0pt, font={bf,footnotesize}}


%%% extras %%%
\RequirePackage{marvosym} % figuras \Letter \Email 
\usepackage{fancyhdr}     % Headers
\usepackage{epsf}


%%% Beamer style %%%
\usetheme{Warsaw}      % estilos slides
%\usefonttheme[onlylarge]{structurebold}   % fontes em negrito
\setbeamertemplate{navigation symbols}{}  % barra de navegacao superior
\setbeamerfont*{frametitle}{size=\normalsize,series=\bfseries} %define tamanhos de letras
\setbeamertemplate{note page}{plain}      % n�o sei
\setbeamercovered{transparent}

%-%-%-%-%-%-%-%-%-%
%  Inicio Slides  %
%-%-%-%-%-%-%-%-%-%

%%% CAPA %%%
\title{Utiliza��o de primitivas no Zookeeper}                                
\author[Douglas A.,Marcelo M.,Tiago S.]{
  Douglas Alves \\
  \and Marcelo Matsumoto  \\
  \and Tiago Silva  
}
\institute{Universidade Estadual de Campinas}
\date{\today}
%%% END CAPA %%%


\begin{document}


%-%-%-%-%-%-%
%  Capa     %
%  Topicos  %
%-%-%-%-%-%-%
\begin{frame}
  \titlepage % slide 1 - capa
\end{frame}

\section{Introdu��o}

\begin{frame}
  \frametitle{Como funciona?}
  \begin{itemize}
 \item Um cliente que quer ser o l�der deve criar um znode ef�mero e sequencial em "/ELECTION". 
\item O cliente cujo znode possuir o menor ID � nomeado o l�der, enquanto os outros clientes esperam enquanto esse n�o caia. 
\item Caso o l�der caia uma nova elei��o � realizada.
\item  Para que a queda seja conhecida, � colocado um watcher sobre o znode do l�der que � disparado quando ele cai. 
  \end{itemize}
\end{frame}

\section{C�digos}


\begin{frame}[fragile=singleslide]
  \frametitle{L�der caiu?}
  \begin{exampleblock}{Election.java}
    \begin{lstlisting}[basicstyle=\tiny]

 synchronized public void process(WatchedEvent event) {
        synchronized (mutex) {
	    if(event.getType() != Event.EventType.None) {
		boolean isNodeDeleted = event.getType().equals(EventType.NodeDeleted);          // Verifica se o evento eh de queda.
		boolean LiderAtual = event.getPath().equals(lider); //Verifica se evento ocorreu no l�der.
		
		if (isNodeDeleted &&  LiderAtual) {
		    System.out.println("O no lider caiu");
		    mutex.notify();
		    notifyAll(); //Acorda clientes
		}
	    }
	}
    }
    \end{lstlisting}
  \end{exampleblock}
\end{frame}


\begin{frame}[fragile=singleslide]
  \frametitle{L�der atual � atualizado}
  \begin{exampleblock}{Election.java}
    \begin{lstlisting}
    /* Insere na classe o lider atual calculado pela classe queue. */
    public void SetLider(String s) {
	this.lider = s;
    }

    \end{lstlisting}
  \end{exampleblock}

\end{frame}

\begin{frame}[fragile=singleslide]
  \frametitle{Esperando l�der cair}
  \begin{exampleblock}{Election.java}
    \begin{lstlisting}
    public void run() {
        try {
            synchronized (this) {
		wait(); // Espera lider cair
            }
        } catch (InterruptedException e) {
        }
    }
    \end{lstlisting}
  \end{exampleblock}
\end{frame}

\begin{frame}[fragile=singleslide]
  \frametitle{Inserindo znode ef�mero na disputa pela lideran�a}
  \begin{exampleblock}{Election.java}
    \begin{lstlisting}
    	String produce(int i) throws KeeperException, InterruptedException{
	    ByteBuffer b = ByteBuffer.allocate(4);
	    byte[] value;

	    // Add child with value i
	    b.putInt(i);
	    value = b.array();
	    return zk.create(root + "/n_", value, Ids.OPEN_ACL_UNSAFE,
			     CreateMode.EPHEMERAL_SEQUENTIAL);
	}


    \end{lstlisting}
  \end{exampleblock}
\end{frame}


\begin{frame}[fragile=singleslide]
  \frametitle{Quantos clientes est�o querendo ser l�der}
  \begin{exampleblock}{Election.java}
    \begin{lstlisting}
	int tamanho() throws KeeperException, InterruptedException{
	    Stat stat = new Stat();
	    zk.getData("/ELECTION", false, stat);
	    
	    return stat.getNumChildren();
	}
    \end{lstlisting}
  \end{exampleblock}
\end{frame}


\begin{frame}[fragile=singleslide]
  \frametitle{Qual � o menor ID?}
  \begin{exampleblock}{Election.java}
    \begin{lstlisting}[basicstyle=\tiny]
	String menor() throws KeeperException, InterruptedException{
	    int retvalue = -1;
	    Stat stat = null;
	    String aux = new String();
 
	    while (true) {
		List<String> list = zk.getChildren(root, true);
		if(list.size() != 0){
		    Integer min = new Integer(list.get(0).substring(7));
		    aux = list.get(0).substring(2);
		    for(String s : list){
			Integer tempValue = new Integer(s.substring(7));
			if(tempValue < min){
			    min = tempValue;
			    aux = s.substring(2);
			}
		    }
		    return aux;
		}
		return INFINITO;
	    }
	}

    \end{lstlisting}
  \end{exampleblock}
\end{frame}


\begin{frame}[fragile=singleslide]
  \frametitle{Monitora��o do znode}
  \begin{exampleblock}{Election.java}
    \begin{lstlisting}
	void monitora(String s){
	    this.dm = new DataMonitor(zk, "/ELECTION/n_" + s, null, this);
	}
    }
    \end{lstlisting}
  \end{exampleblock}
\end{frame}


\begin{frame}[fragile=singleslide]
  \frametitle{Cria��o de znode e espera pela queda do l�der}
  \begin{exampleblock}{Election.java}
    \begin{lstlisting}[basicstyle=\tiny]
    public static void election(String args[]){
	Queue q = new Queue(args[0], "/ELECTION");
	try{
	    int selfId = Integer.parseInt(q.produce(0).substring(13)); //Criacao do znode
	    System.out.println("Meu ID: " + selfId);

	    List<String> list = q.zk.getChildren("/ELECTION", true);
	    int aux = -1;
	    while (q.tamanho() != 0) {
		Integer menor = new Integer(q.menor());
		if(selfId < menor){
		    System.out.println("Eu morri!!");
		    return;
		}
		if(menor != aux){
		    System.out.println("Menor filho:" + menor);
		    aux = menor;
		    q.SetLider( "/ELECTION/n_" + q.menor());
		    if(selfId == menor){ //Verifica se o menor id eh o meu.
			System.out.println("Eu sou o lider\n");
		    }else{
			naoLider(q);
		    }
    \end{lstlisting}
  \end{exampleblock}
\end{frame}


\begin{frame}[fragile=singleslide]
  \frametitle{Fun��o n�o l�der}
  \begin{exampleblock}{Election.java}
    \begin{lstlisting}

 static public void naoLider(Queue q){
	try{
	    System.out.println("O lider nao sou eu");
	    System.out.println("Entao eu vou monitorar o lider...\n");
	    q.monitora(q.menor());
	    q.run();
	    System.out.println("Fique esperando algo acontecer...\n");
	} catch (KeeperException e){
	} catch (InterruptedException e){
	}
	return;
    }
    \end{lstlisting}
  \end{exampleblock}
\end{frame}

\end{document}
