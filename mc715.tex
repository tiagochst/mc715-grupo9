%-%-%-%-%-%-%-%-%-%-%-%-%-%-%-%-%-%-%-%-%-%-%-%-%-%
%  MC514 Sistemas Operacionais: Teoria e Pr�tica  %  
%  Projeto Final: Inicializa��o do Linux (Boot)   %
%  Data:26/06/2010                                %
%  Unicamp,Campinas,S�o Paulo,Brasil              %
%  Grupo:                                         %
%    - Alexandre Nobuo Kunieda                    %
%    - Daniel Catarino Biscalchin                 %
%    - Daniel Naoyoshi Fukuciro Parrode           %
%    - Tiago Chedraoui Silva                      %
%-%-%-%-%-%-%-%-%-%-%-%-%-%-%-%-%-%-%-%-%-%-%-%-%-%
\documentclass[dvips,11pt]{beamer}

%%% fontes %%%
\usepackage[T1]{fontenc}
\usepackage[latin9]{inputenc}
\usepackage[brazil]{babel}    % d� suporte para os termos na l�ngua portuguesa do Brasi


%%% matematicos %%%
\usepackage{amsmath}
\usepackage{amssymb}
\usepackage{mathptmx}

\usepackage{multicol}  
%%% figuras %%%
\usepackage{graphicx}
\usepackage{wrapfig}


%%% tabelas %%%
\usepackage{colortbl}
\usepackage{array}
\usepackage{longtable}
\usepackage{fancyvrb}
\usepackage{color}

%%% outros %%%
\usepackage{url}
\usepackage{textcomp}
\usepackage{hyperref} %internal links
\usepackage{color}       
\usepackage{indentfirst} %retira padrao americano de paragrafos
\usepackage{multicol}    
\numberwithin{table}{section}
\numberwithin{figure}{section} %numercao de figuras por secao


%%% usado para codigo %%%
\usepackage{listings}
\lstset{
  language = C,
  basicstyle=\footnotesize\ttfamily, 
  numbers=left,               
  numberstyle=\tiny,         
  %stepnumber=2,              
  numbersep=5pt,             
  tabsize=2,                  
  extendedchars=true,  
  breaklines=true,       
  keywordstyle=\color{blue},
  frame=b,         
  stringstyle=\color{green}\ttfamily, 
  showspaces=false,          
  showtabs=false,             
  xleftmargin=17pt,
  framexleftmargin=17pt,
  framexrightmargin=5pt,
  framexbottommargin=4pt,
  %backgroundcolor=\color{lightgray},
  showstringspaces=false              
}
\lstloadlanguages{C}
\usepackage{caption}
\DeclareCaptionFont{white}{\color{white}}
\DeclareCaptionFont{green}{\color{green}}
\DeclareCaptionFormat{listing}{\colorbox[cmyk]{0.43, 0.35, 0.35,0.01}{\parbox{\textwidth}{\hspace{15pt}#1#2#3}}}
\captionsetup[lstlisting]{format=listing,labelfont=white,textfont=white, singlelinecheck=false, margin=0pt, font={bf,footnotesize}}


%%% extras %%%
\RequirePackage{marvosym} % figuras \Letter \Email 
\usepackage{fancyhdr}     % Headers
\usepackage{epsf}


%%% Beamer style %%%
\usetheme{Warsaw}      % estilos slides
%\usefonttheme[onlylarge]{structurebold}   % fontes em negrito
\setbeamertemplate{navigation symbols}{}  % barra de navegacao superior
\setbeamerfont*{frametitle}{size=\normalsize,series=\bfseries} %define tamanhos de letras
\setbeamertemplate{note page}{plain}      % n�o sei
\setbeamercovered{transparent}

%-%-%-%-%-%-%-%-%-%
%  Inicio Slides  %
%-%-%-%-%-%-%-%-%-%

%%% CAPA %%%
\title{Utiliza��o de primitivas no Zookeeper}                                
\author[Douglas A.,Marcelo M.,Tiago S.]{
  Douglas Alves \\
  \and Marcelo Matsumoto  \\
  \and Tiago Silva  
}
\institute{Universidade Estadual de Campinas}
\date{\today}
%%% END CAPA %%%


\begin{document}


%-%-%-%-%-%-%
%  Capa     %
%  Topicos  %
%-%-%-%-%-%-%
\begin{frame}
  \titlepage % slide 1 - capa
\end{frame}

\begin{frame}
  \frametitle{T�picos}

  \tableofcontents%[pausesections]

\end{frame}


%-%-%-%-%-%-%-%-%-%-%-%
%  Introdu��o         %
%   - O que � o boot  %
%   - Utilidades      %
%   - Dificuldades    %
%-%-%-%-%-%-%-%-%-%-%-%
\section{Introdu��o}


\begin{frame}[fragile=singleslide]
  \frametitle{Configura��es }
  \begin{exampleblock}{conf/node01.cfg}
    \begin{lstlisting}
# The number of milliseconds of each tick 
tickTime=2000
# The number of ticks that the initial
# synchronization phase can take
initLimit=10
# The number of ticks that can pass between
# sending a request and getting an acknowledgement
syncLimit=5
# the directory where the snapshot is stored.
dataDir=/home/cluster/mc715u47/zookeeper-3.3.2/data/node01
# the port at which the clients will connect
clientPort=33999

server.1=node01:33998:33997
server.3=node03:33998:33997
server.5=node05:33998:33997
    \end{lstlisting}
  \end{exampleblock}

  %try out     dmesg | less 

\end{frame}



\begin{frame}[fragile]

\frametitle{verbatim}
\begin{verbatim}
for i in range(1, 5):
  print i
else:
  print "The for loop is over"
\end{verbatim}
\end{frame}

\begin{frame}[fragile]
\frametitle{Exemplo de fila em python}
\newcommand\codeHighlight[1]{\textcolor[rgb]{1,0,0}{\textbf{#1}}}
\begin{Verbatim}[commandchars=\\\{\}]
\$ python queue.py
Connected
Queue already exists
Enqueuing 10 items
Numero de filhos: 1
Numero de filhos: 2
Numero de filhos: 3
Numero de filhos: 4
Numero de filhos: 5
Numero de filhos: 6
Numero de filhos: 7
Numero de filhos: 8
Numero de filhos: 9
Numero de filhos: 10
Done
\end{Verbatim}
\end{frame}




\begin{frame}[fragile]
\frametitle{Fila - N� 3}
\newcommand\codeHighlight[1]{\textcolor[rgb]{1,0,0}{\textbf{#1}}}
\codeHighlight{N� 3 desenfilera 8 itens do n� 5. (todos, menos os itens de n�mero 4 e 9)}
 \begin{columns}
      \begin{column}{.5\linewidth}
\begin{Verbatim}[commandchars=\\\{\}]
Thread 0: queue item 0
Numero de filhos: -1
Thread 1: queue item 1
Numero de filhos: -2
Thread 2: queue item 2
Numero de filhos: -3
Thread 3: queue item 3
Numero de filhos: -4
\end{Verbatim}

  \end{column}
      \begin{column}{.5\linewidth}
\begin{Verbatim}[commandchars=\\\{\}]

Thread 4: queue item 5
Numero de filhos: -5
Thread 0: queue item 6
Numero de filhos: -6
Thread 1: queue item 7
Numero de filhos: -7
Thread 2: queue item 8
Numero de filhos: -8
Done 
\end{Verbatim}


     \end{column}
    \end{columns}
\end{frame}




\begin{frame}[fragile]
\frametitle{Fila - N� 5}
\newcommand\codeHighlight[1]{\textcolor[rgb]{1,0,0}{\textbf{#1}}}
\codeHighlight{N� 5 desenfileira somente 2 itens, 4 e 9.}
\begin{Verbatim}[commandchars=\\\{\}]
Consuming all items in queue with 5 threads
Thread 0: queue item 4
Numero de filhos: 9
Thread 0: queue item 9
Numero de filhos: 8
Done
\end{Verbatim}
\end{frame}












\begin{frame}[fragile]
\frametitle{Fila - N� 3}
\newcommand\codeHighlight[1]{\textcolor[rgb]{1,0,0}{\textbf{#1}}}
%\codeHighlight{N� 3 desenfilera 8 itens do n� 5. (todos, menos os itens de n�mero 4 e 9)}
 \begin{columns}
      \begin{column}{.5\linewidth}
\begin{Verbatim}[commandchars=\\\{\}]
Consuming all items in queue with 5 threads
Thread 0: queue item 0
Numero de filhos: 9
Thread 2: queue item 2
Numero de filhos: 8
Thread 1: queue item 4
Numero de filhos: 7
Thread 3: queue item 6
Numero de filhos: 6
Thread 4: queue item 8
Numero de filhos: 5
\end{Verbatim}

  \end{column}
      \begin{column}{.5\linewidth}
\begin{Verbatim}[commandchars=\\\{\}]

Thread 0: queue item 1
Numero de filhos: 4
Thread 1: queue item 5
Numero de filhos: 3
Thread 2: queue item 3
Numero de filhos: 2
Thread 3: queue item 7
Numero de filhos: 1
Thread 4: queue item 9
Numero de filhos: 0 
\end{Verbatim}


     \end{column}
    \end{columns}
\end{frame}


\begin{frame}[fragile=singleslide]
  \frametitle{Inserindo notifica��es no kernel}
  \begin{exampleblock}{linux-2.6.22.19/init/main.c}
    \begin{lstlisting}
  printk(KERN_WARNING "I am a kernel hacker!\n");
  /* Fun��o que imprime mensagem
     durante o processo de boot...YES!! */
    \end{lstlisting}
  \end{exampleblock}

  %try out     dmesg | less 

\end{frame}
\end{document}
